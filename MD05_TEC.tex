\documentclass[xcolor={svgnames},
  hyperref={colorlinks},
  t,
  spanish, 12pt]{beamer}
\mode<presentation>

\usefonttheme[onlymath]{serif}
\setbeamertemplate{theorems}[ams style]
\usetheme{metropolis}
\useinnertheme{rectangles}
\usecolortheme{dolphin}

\setbeamercovered{highly dynamic}

\newcounter{saveenumi}
\newcommand{\seti}{\setcounter{saveenumi}{\value{enumi}}}
\newcommand{\conti}{\setcounter{enumi}{\value{saveenumi}}}

\resetcounteronoverlays{saveenumi}

\setbeamercolor{emph}{fg=purple}
\renewcommand<>{\emph}[1]{%

  {\usebeamercolor[fg]{emph}\only#2{\itshape}#1}%
}

\usepackage{fontenc}
\usepackage{graphicx}
\usepackage[utf8]{inputenc}
\usepackage[spanish,mexico]{babel}
\usepackage{fontenc}
\usepackage{amsmath}
\usepackage{amsthm}
\usepackage{amssymb}
\usepackage{graphicx}
\usepackage{mathrsfs}
\usepackage{yfonts}
\usepackage{enumerate}
\usepackage{mathtools}
\usepackage{textcomp}
\usepackage{lmodern}
\usepackage{fancyvrb}
\usepackage{multicol}
\usepackage{color}
\usepackage{verbatim}

\DefineVerbatimEnvironment{ColorVerbatim}{Verbatim}%
{formatcom=\color{purple},commandchars=\\\{\}}

\usepackage{etoolbox}

\BeforeBeginEnvironment{Verbatim}{\begingroup\color{purple}}%
\AfterEndEnvironment{Verbatim}{\endgroup}
\numberwithin{equation}{section}
\numberwithin{figure}{section}

\def\subsubsectionname{\translate{}}
\def\insertsubsubsectionnumber{\arabic{subsubsection}}
\setbeamertemplate{subsubsection page}
{
  \begin{centering}
    {\usebeamerfont{subsubsection name}\usebeamercolor[fg]{subsubsection
        name}\subsubsectionname}
    \vskip1em\par
    \begin{beamercolorbox}[sep=4pt,center]{part title}
      \usebeamerfont{subsubsection title}\insertsubsubsection\par
    \end{beamercolorbox}
  \end{centering}
}
\def\subsubsectionpage{\usebeamertemplate*{subsubsection page}}

\AtBeginSection{\frame{\sectionpage}}
\AtBeginSubsection{\frame{\subsectionpage}}
\AtBeginSubsubsection{\frame{\subsubsectionpage}}

\input{./00-03-entorno.tex}
\input{./00-01-comandos.tex}

%\date{\today}
\title{Matem\'aticas Discretas \\
  Teoría de Gr\'aficas}
\author[Juliho Castillo]{Dr. Juliho Castillo}

\date{\today}

\begin{document}

\frame{
  \titlepage
}

\begin{frame}[allowframebreaks=0.5]
  \tableofcontents
\end{frame}

\section{Matrices}

\begin{frame}
  Las matrices son arreglos rectangulares de n\'umero que nos ayudan a codificar
  informaci\'on. Por ejemplo:
  $$
    \begin{pmatrix}
      a_{1,1} & a_{1,2} \\
      a_{2,1} & a_{2,2}
    \end{pmatrix}
  $$
  puede ser \'util para codificar los coeficientes del sistema de ecuaciones:
  $$
    \begin{cases}
      a_{1,1}x+a_{1,2}y=b_{1} \\
      a_{2,1}x+a_{2,2}y=b_{2}
    \end{cases}
  $$
\end{frame}

\begin{frame}
  En general, una matriz tiene la forma
  \begin{equation}
    \label{A}
    \tag{A}
    \begin{pmatrix}
      a_{1,1} & a_{1,2} & \cdots & a_{1,n} \\
      \vdots  &         &        & \vdots  \\
      a_{m,1} & a_{m,2} & \cdots & a_{m,n}
    \end{pmatrix}
  \end{equation} \pause

  Los subíndices de cada elemento $a_{i,j}$ denotan la posici\'on del mismo:
  $i$ es el n\'umero del \emph{rengl\'on} (contando de arriba a abajo), mientras
  que $j$ es el n\'umero de la columna (contanto de izquierda a derecha).
\end{frame}

\begin{frame}
  Podemos extraer renglones y columnas de la matrix \eqref{A}: El $i-$esímo
  rengl\'on es
  $$
    R_{i}=
    \begin{pmatrix}
      a_{i,1} & \cdots & a_{i,n}
    \end{pmatrix}
  $$ \pause
  mientras que la $j-$\'esima columna ser\'a
  $$
    C_{j}=
    \begin{pmatrix}
      a_{j,1} \\
      \vdots  \\
      a_{j,m}
    \end{pmatrix}
  $$
\end{frame}

\begin{frame}
  Diremos que la matriz \eqref{A} tiene dimensi\'on $m\times n.$ \pause

  Si existe un conjunto de n\'umeros $F,$ tal que todos los elementos $a_{i,j}$
  de la matriz pertenecen a dicho conjunto, diremos que la matriz tiene
  coeficientes en $F.$ \pause
\end{frame}

\begin{frame}
  \begin{rem}
    Para que las operaciones entre matrices est\'en bien definidas, es necesario
    que la suma, resta y multiplicaci\'on entre entre elementos de $F$ tambi\'en
    este bien definida. Por esto generalmente $F$ se elige como $\R$ o $\Z.$
  \end{rem}
\end{frame}

\begin{frame}
  La colecci\'on de todas las matrices de dimensi\'on $m\times n$ con
  coeficientes en $F$ se denotar\'a por $$M_{m,n}(F).$$
\end{frame}

\begin{frame}
  \begin{defn}
    Las matrices de dimensi\'on $m\times 1$ se conocen como \emph{vectores
      columna,} mientras que las de dimensi\'on $1\times n$ se conocen como
    \emph{vectores rengl\'on.}
    \pause

    La colecci\'on $M_{m,1}(F)$ de todos los vectores columna con coeficientes
    comunmente se denota por $F^{m}.$ \pause Mientras que la colecci\'on
    $M_{1,n}(F)$ de todos los vectores columna con coeficientes  comunmente se
    denota por $F^{n\ast}.$

  \end{defn}
\end{frame}

\subsection{Operaciones elementales}

\begin{frame}
  Por brevedad, la matriz \eqref{A} se denota por $A=[a_{i,j}].$ \pause

  En el caso de los vectores renglones y columnas, podemos omitir el subíndice
  fijo
  $$R=[R_{1,j}]=[R_{j}], \; C=[C_{i,1}]=[C_{i}].$$
\end{frame}

\begin{frame}
  Si $B=[b_{i,j}]$ es otra matriz de dimensi\'on $m\times n,$ la suma se define
  como $$A+B=[a_{i.j}+b_{i,j}].$$\pause

  De manera similar, la resta se define como $$A-B=[a_{i,j}-b_{i,j}].$$
\end{frame}

\begin{frame}
  \begin{exmp}
    $$
      \begin{pmatrix}
        1 & -1 & 0  \\
        2 & 3  & -4
      \end{pmatrix}
      +
      \begin{pmatrix}
        7 & 0  & -1 \\
        2 & -1 & 5
      \end{pmatrix} =
    $$ \pause
    $$
      \begin{pmatrix}
        1 & -1 & 0  \\
        2 & 3  & -4
      \end{pmatrix}
      -
      \begin{pmatrix}
        7 & 0  & -1 \\
        2 & -1 & 5
      \end{pmatrix} =
    $$
  \end{exmp}

\end{frame}

\begin{frame}
  Observe que para que la \emph{suma y resta} tenga sentido, ambas matrices deben
  tener exactamente las \emph{mismas dimensiones}. \pause

  Despu\'es de ver la facilidad para definir la suma y resta, uno se ve tentado a
  definir la multiplicaci\'on de la misma forma. Pero tal definici\'on es poco
  \'util en las aplicaciones. \pause

  Por esta raz\'on, desarrollaremos el concepto de multiplicaci\'on, a fin de
  poder aplicar esta operaci\'on en la resoluci\'on de problemas.
\end{frame}

\subsection{Multiplicaci\'on}
\begin{frame}
  \begin{defn}
    Sean $R=[R_{j}]$ un vector rengl\'on y $C=[C_{i}]$ un vector columna, ambos de
    longitud $n.$
    \pause
    El \emph{producto rengl\'on-columna} se define como
    \begin{equation}
      \label{RC}
      \tag{RC}
      RC=
      \begin{pmatrix}
        R_{1} & \cdots & R_{n}
      \end{pmatrix}
      \begin{pmatrix}
        C_{1} \\ \vdots \\ C_{n}
      \end{pmatrix}
      =
      \sum_{i=1}^{n} R_{j}C_{i}.
    \end{equation}

  \end{defn}

\end{frame}

\begin{frame}
  \begin{exmp}
    Considere
    $$
      R=
      \begin{pmatrix}
        1 & 0 & -1
      \end{pmatrix}, \;
      C=
      \begin{pmatrix}
        2 \\ 1 \\ -2
      \end{pmatrix}.
    $$

    Calcule $RC.$

  \end{exmp}

\end{frame}

\begin{frame}
  \begin{exmp}
    Reescriba la siguiente ecuaci\'on, utilizando el \emph{producto
      rengl\'on-columna}:
    $$2x-3y+z=0.$$
  \end{exmp}

\end{frame}

\begin{frame}
  \begin{defn}
    Sea $A=[a_{i,j}]\in M_{m\times n}$ y $B=[b_{j,k}]\in M_{n\times l}.$ Definimos
    su producto como
    \begin{equation}
      \label{AB}
      \tag{AB}
      AB=
      \begin{pmatrix}
        R_{i}C_{k}
      \end{pmatrix}
    \end{equation}
    donde $R_{i}$ es el $i-$\'esimo rengl\'on de $A$ y $C_{k}$ es la $k-$\'esima
    columna de $B.$
  \end{defn}

\end{frame}

\begin{frame}
  \begin{rem}
    \begin{itemize}
      \item Para que esta multiplicaci\'on tenga sentido, los renglones de $A$ y
            las columnas de $B$ deber\'an tener la misma longitud $n.$
            \pause
      \item La matriz resultante tendr\'a dimensi\'on $m \times l.$ \pause
      \item A menos que $m=l,$ el producto $BA$ podría no estar definido. \pause
      \item Aun cuando $BA$ estuviera bien definido, el producto de matrices no es
            \emph{conmutativo,} es decir, generalmente tendremos que $$AB \neq BA.$$
    \end{itemize}
  \end{rem}

\end{frame}

\begin{frame}
  \begin{exmp}
    Encuentre el producto $AB$ de las siguientes matrices
    $$A= \left(\begin{array}{r}
        0
      \end{array}\right) $$
    $$B= \left(\begin{array}{rr}
        0 & -1
      \end{array}\right) $$
    \pause Solución:
    $$AB= \left(\begin{array}{rr}
        0 & 0
      \end{array}\right) $$
  \end{exmp}
\end{frame}

\begin{frame}
  \begin{exmp}
    Encuentre el producto $AB$ de las siguientes matrices
    $$A= \left(\begin{array}{rr}
        0  & -1 \\
        -1 & 0  \\
        0  & 0
      \end{array}\right) $$
    $$B= \left(\begin{array}{rrr}
        0 & -1 & 0 \\
        0 & 0  & 0
      \end{array}\right) $$
    \pause Solución:
    $$AB= \left(\begin{array}{rrr}
        0 & 0 & 0 \\
        0 & 1 & 0 \\
        0 & 0 & 0
      \end{array}\right) $$
  \end{exmp}
\end{frame}

\begin{frame}
  \begin{exmp}
    Encuentre el producto $AB$ de las siguientes matrices
    $$A= \left(\begin{array}{r}
        -1 \\
        -1 \\
        0
      \end{array}\right) $$
    $$B= \left(\begin{array}{rrr}
        -1 & 0 & 0
      \end{array}\right) $$
    \pause Solución:
    $$AB= \left(\begin{array}{rrr}
        1 & 0 & 0 \\
        1 & 0 & 0 \\
        0 & 0 & 0
      \end{array}\right) $$
  \end{exmp}
\end{frame}

\begin{frame}
  \begin{exmp}
    Encuentre el producto $AB$ de las siguientes matrices
    $$A= \left(\begin{array}{r}
        6  \\
        -9 \\
        -10
      \end{array}\right) $$
    $$B= \left(\begin{array}{r}
        -5
      \end{array}\right) $$
    \pause Solución:
    $$AB= \left(\begin{array}{r}
        -30 \\
        45  \\
        50
      \end{array}\right) $$
  \end{exmp}
\end{frame}

\begin{frame}
  \begin{exmp}
    Encuentre el producto $AB$ de las siguientes matrices
    $$A= \left(\begin{array}{r}
        2
      \end{array}\right) $$
    $$B= \left(\begin{array}{rrr}
        -1 & 1 & -3
      \end{array}\right) $$
    \pause Solución:
    $$AB= \left(\begin{array}{rrr}
        -2 & 2 & -6
      \end{array}\right) $$
  \end{exmp}
\end{frame}

\begin{frame}
  \begin{exmp}
    Encuentre el producto $AB$ de las siguientes matrices
    $$A= \left(\begin{array}{rr}
        -1 & -3 \\
        -7 & -1
      \end{array}\right) $$
    $$B= \left(\begin{array}{r}
        -7 \\
        -4
      \end{array}\right) $$
    \pause Solución:
    $$AB= \left(\begin{array}{r}
        19 \\
        53
      \end{array}\right) $$
  \end{exmp}
\end{frame}

\begin{frame}
  \begin{exmp}
    Rescriba el siguiente sistema de ecuaci\'on en forma matricial y encuentre su
    soluci\'on:
    $$
      \begin{cases}
        -x-3y=19 \\
        -7x-y=53
      \end{cases}
    $$
  \end{exmp}

\end{frame}

\section{Teoría general de grafos}

\begin{frame}
  En matem\'aticas, la \emph{teoría de grafos} estudia estructuras
  matem\'aticas usadas para modelar relaciones por pares entre objetos.
\end{frame}

\subsection{Definici\'on de grafo}

\begin{frame}{Concepto de gr\'afo}
  Un \emph{grafo} $G$ consiste de:
  \begin{enumerate}[(a)]
    \item Un conjunto $V$ cuyos elementos son llamados \emph{v\'ertices,} puntos o
          nodos de $G.$
    \item Un conjunto $E$ de pares (no ordenados) de distintos vertices, a los que
          llamaremos \emph{aristas} de $G.$
  \end{enumerate}

  Denotaremos un grafo por $G(V,E)$ cuando querramos enfatizar los componentes
  del mismo.
\end{frame}

\begin{frame}
  \begin{rem}
    Debido a una ambig\"uedad en la traducci\'on del ingl\'es al espa\~nol, en
    ocasiones, a un grafo tambi\'en se le conoce como \emph{gr\'afica,} que se
    puede confundir con el concepto de teoría de conjuntos. En este material, a
    veces utilizaremos \textit{gr\'afica,} pero debe entenderse como un grafo.
  \end{rem}
\end{frame}

\begin{frame}
  \begin{figure}[h!]
    \centering
    \includegraphics[width=12cm,keepaspectratio=true]{./grafos.png}
    % grafos.png: 0x0 pixel, 300dpi, 0.00x0.00 cm, bb=
    \caption{Grafos y multigrafos}
    \label{fig:md0501}
  \end{figure}

\end{frame}

\begin{frame}{Multigrafos}
  Consideremos la figura \ref{fig:md0501} (b). Las aristas $e_{4}$y $e_{5}$ son
  llamadas \emph{aristas multiples} ya que conectan los mismos extremos, \pause
  mientras que la arista $e_{6}$ es llamada \emph{bucle} ya que conecta un
  v\'ertice consigo mismo.

\end{frame}

\begin{frame}
  Tales diagramas son llamados \emph{multigrafos;} \pause la definici\'on formal
  de grafo no admite aristas multiples ni bucles.
\end{frame}

\begin{frame}
  \begin{rem}
    Sin embargo, algunos textos utilizan ``grafos'' para referirse a lo que
    nosotros llamaremos multigrafos, mientras que ocupan ``grafo simple'' para lo
    que nosotros llamaremos grafos.
  \end{rem}

\end{frame}

\begin{frame}{Grado de un v\'ertice}
  El \emph{grado} de un v\'ertice $v$ es un grafo $G,$ denotado por $\deg(v),$ es
  igual al n\'umero de aristas in $G$ que contienen a $v,$ es decir, que
  \emph{inciden} en $v.$
\end{frame}

\begin{frame}
  Dado que cada arista incide en dos v\'ertices diferentes, tenemos el siguiente
  resultado simple pero importante:
  \begin{thm}
    La suma de los grados de los v\'ertices en un grafo $G$ es el doble del
    n\'umero de aristas.
  \end{thm}

\end{frame}

\begin{frame}
  \begin{exmp}
    En el grafo de la figura \ref{fig:md0501}(a), tenemos que
    $$\deg(A)=2, \; \deg(B)=3,\; \deg(C)=3, \; \deg(D)=2.$$
    \pause

    La suma de los grados es igual a 10, que es dos veces el n\'umero de aristas.
  \end{exmp}

\end{frame}

\begin{frame}
  \begin{defn}
    Diremos que un v\'ertice es \emph{par} o \emph{impar} de acuerdo a la paridad
    de su grado.
    \pause
    En el ejemplo anterior, tanto $A$ com $D$ son v\'ertices pares, mientras que
    $B$ y $C$ son impares.
  \end{defn}

\end{frame}

\begin{frame}
  \begin{rem}
    Diremos que un vertice de grado cero est\'a \emph{aislado.}
  \end{rem}
\end{frame}

\begin{frame}{Gr\'afos finitos y triviales}
  Diremos que un grafo  es \emph{finito} si tiene un n\'umero finito de
  v\'ertices y un n\'umero finito de aristas. \pause

  Observe que un n\'umero finito de v\'ertices implica un n\'umero finito de
  aristas; pero no lo contrario no es necesariamente cierto. \pause

  Diremos que un grafo con un \'unico v\'ertice, sin aristas,  es
  \emph{trivial.}\
\end{frame}

\begin{frame}
  \begin{rem}
    A menos que se indique de otra manera, s\'olo trataremos con grafos finitos.
  \end{rem}

\end{frame}

\subsection{Subgrafos y grafos homeomorfos e isomorfos}

\begin{frame}
  Ahora, discutiremos relaciones de equivalencia entre grafos.
\end{frame}

\begin{frame}{Subgrafos}
  Consideremos un grafo $G(V,E).$ Diremos que otro grafo $H(V',E')$ es un
  \emph{subgrafo} de $G$ si los v\'ertices y aristas de $H$ est\'an contenidos en
  los v\'ertices y aristas de $G,$ es decir,
  $$
    V'\subset V, \; E' \subset E.
  $$
\end{frame}

\begin{frame}
  En particular:
  \begin{enumerate}[(a)]
    \item Un subgrafo $H(V',E')$ de $G(V,E)$ es llamado subgrafo \emph{inducido}
          por sus v\'ertices $V'$ si el conjunto de aristas $E'$ contiene todas las
          aristas en $G$ cuyo extremos pertenecen a los v\'ertices en $H.$ \pause
    \item Si $v$ es un v\'ertice en $G,$ entonces \emph{$G-v$} es el subgrafo de
          $G$ ontenido al borrar $v$ de $G$ y todas las aristas en $G$ que inciden en
          $v.$ \pause
    \item Si $e$ es una arista en $G,$ entonces \emph{$G-e$} es el subgrafo de $G$
          obtenido borrando la arista $e$ en $G.$
  \end{enumerate}

\end{frame}

\begin{frame}{Grafos isomorfos}
  Dos grafos $G(V,E)$ y $G^{*}(V^{*},E^{*})$ son llamados \emph{isomorfos} si
  existe una funci\'on biyectiva $f: V \to V^{*}$ tal que: $\set{u,v}$ es una
  arista de $G$ si y solo si $\set{f(u),f(v)}$ es una arista de $G^{*}.$
  \pause

  La idea es que estos grafos son equivalentes, a\'un cuando sus representaciones
  pueden lucir muy diferentes.
\end{frame}

\begin{frame}
  \begin{figure}[h!]
    \centering
    \includegraphics[width=10cm,keepaspectratio=true]{./letras.png}
    % letras.png: 0x0 pixel, 300dpi, 0.00x0.00 cm, bb=
    \caption{Grafos isomorfos.}
    \label{fig:md0502}
  \end{figure}

\end{frame}

\begin{frame}{Grafos homeomorfos}
  Dado un grafo $G,$ podemos obtener un nuevo grafo dividiendo una arista de $G$
  con v\'ertices adicionales. \pause

  Dos grafos $G$ y $G^{*}$ son llamados \emph{homeomorfos} si pueden obtenerse de
  gr\'aficas isomorfas a trav\'es de este m\'etodo.
\end{frame}

\begin{frame}
  \begin{figure}[h!]
    \centering
    \includegraphics[width=8cm,keepaspectratio=true]{./homomorfas.png}
    % homomorfas.png: 0x0 pixel, 300dpi, 0.00x0.00 cm, bb=
    \caption{Grafos homomorfos}
    \label{fig:md0503}
  \end{figure}

  Los grafos $(a)$ y $(b)$ son homeomorfos, ya que se pueden obtener a\~nadiendo
  v\'ertices al grafo $(c).$
\end{frame}

\subsection{Caminos y conexidad}

\begin{frame}
  Un \emph{camino} en un (multi)grafo $G$ consiste en una sucesi\'on alternante
  de v\'ertices y arista de la forma
  $$
    v_{0}, e_{1}, v_{1}, ..., e_{n-1}, v_{n-1}, e_{n}, v_{n}
  $$
  donde cada arista $e_{i}$ contiene los v\'ertices $v_{i-1}$ y $v_{i}.$
\end{frame}

\begin{frame}
  \begin{rem}
    Observe que en grafo, podemos simplificar la notaci\'on para un camino,
    indicando s\'olo los v\'ertices que recorre:
    $$v_{0}, v_{1},..., v_{n}.$$
  \end{rem}
\end{frame}

\begin{frame}
  Diremos que el camino es \emph{cerrado} si $v_{n}=v_{0}.$ En otro caso, diremos
  que el camino conecta $v_{0}$ con $v_{n}.$
  \pause

  Un \emph{camino simple} es aquel en el cual todos los v\'ertices son distintos.
  Mientras que un camino en el que todas las aristas son distintas se llama
  \emph{paseo}.
  \pause
\end{frame}

\begin{frame}
  La \emph{longitud} de un camino es igual a n\'umero de aristas en la sucesi\'on
  que lo define.
  \pause
\end{frame}

\begin{frame}
  Un \emph{ciclo} es un camino cerrado de \emph{longitud} al menos 3, en el que
  todos los v\'ertices son distintos, excepto el inicial $v_{0}$ y el final
  $v_{n}.$
  \pause

  Un ciclo de longitud $k$ es llamado \emph{$k-$ciclo.}
\end{frame}

\begin{frame}
  \begin{figure}[h!]
    \centering
    \includegraphics[width=10 cm,keepaspectratio=true]{./grafo_8_8.png}
    % grafo_8.8.png: 0x0 pixel, 300dpi, 0.00x0.00 cm, bb=
    \caption{Conexidad en grafos}
    \label{fig:md0504}
  \end{figure}
\end{frame}

\begin{frame}
  \begin{exmp}
    \label{lip:exmp:8.1}
    Consideremos el grafo \ref{fig:md0504}(a). Considere las siguientes sucesiones
    \begin{align*}
      \a   & =\left( P_{4}, P_{1}, P_{2}, P_{5}, P_{1},P_{2}, P_{3}, P_{6}  \right), \\
      \b   & =\left( P_{4}, P_{1}, P_{5}, P_{2}, P_{6} \right)                       \\
      \gam & = \left( P_{4}, P_{1}, P_{5}, P_{2}, P_{3}, P_{5}, P_{6} \right)        \\
      \del & =\left( P_{4}, P_{1}, P_{5}, P_{3}, P_{6} \right).
    \end{align*}

  \end{exmp}

\end{frame}

\begin{frame}
  \begin{figure}[h!]
    \centering
    \includegraphics[width=5cm,keepaspectratio=true]{./grafo_8_8_a.png}
    % grafo_8_8_a.png: 0x0 pixel, 300dpi, 0.00x0.00 cm, bb=
  \end{figure}
  $\a$ es un camino de $P_{4}$ a $P_{6},$ pero no es un paseo.
\end{frame}

\begin{frame}
  \begin{figure}[h!]
    \centering
    \includegraphics[width=5cm,keepaspectratio=true]{./grafo_8_8_a.png}
    % grafo_8_8_a.png: 0x0 pixel, 300dpi, 0.00x0.00 cm, bb=
  \end{figure}
  $\b$ no es un camino, ya que no existe alguna arista $\set{P_{2}, P_{6}}.$
\end{frame}

\begin{frame}
  \begin{figure}[h!]
    \centering
    \includegraphics[width=5cm,keepaspectratio=true]{./grafo_8_8_a.png}
    % grafo_8_8_a.png: 0x0 pixel, 300dpi, 0.00x0.00 cm, bb=
  \end{figure}
  $\gam$ es un paseo, pero no es un camino simple.
\end{frame}

\begin{frame}
  \begin{figure}[h!]
    \centering
    \includegraphics[width=5cm,keepaspectratio=true]{./grafo_8_8_a.png}
    % grafo_8_8_a.png: 0x0 pixel, 300dpi, 0.00x0.00 cm, bb=
  \end{figure}
  $\del$ es un camino simple de $P_{4}$ a $P_{6},$ pero no es el camino m\'as
  corto, es decir, con el meno n\'umero de aristas. ?`Cu\'al es el camino m\'as
  corto?
\end{frame}

\begin{frame}
  Eliminando aristas innecesarias, no es difícil ver que cualquier camino de
  $u$ a $v$ puede ser reemplazado por un camino simple.
  \pause

  Formalmente:
  \begin{thm}
    Existe un camino del v\'ertice $u$ a $v$ si y solo si existe un camino simple
    de $u$ a $v.$
  \end{thm}

\end{frame}

\begin{frame}{Conexidad y componentes conexas}
  Un grafo $G$ es conexo si existe un camino entre cualesquiera dos v\'ertices.
  \pause Por ejemplo, el grafo \ref{fig:md0504}(a) es conexo, pero no así el
  grafo \ref{fig:md0504}(b).
  \begin{figure}[h!]
    \centering
    \includegraphics[width=8cm,keepaspectratio=true]{./grafo_8_8.png}
    % grafo_8_8.png: 0x0 pixel, 300dpi, 0.00x0.00 cm, bb=
  \end{figure}

\end{frame}

\begin{frame}
  Consideremos un grafo $G.$ Un subgrafo conexo $H$ de $G$ es llamado
  \emph{componente conexa} de $G$ si $H$ no est\'a contenido de manera propia en
  cualquier otro grafo conexo de $G.$\pause

  Por ejemplo, el grafo \ref{fig:md0504}(b) tiene tres componentes conexas.

  \begin{figure}[h!]
    \centering
    \includegraphics[width=8cm,keepaspectratio=true]{./grafo_8_8.png}
    % grafo_8_8_a.png: 0x0 pixel, 300dpi, 0.00x0.00 cm, bb=
  \end{figure}

\end{frame}

\begin{frame}
  \begin{rem}
    Formalmente, permitiendo que un v\'ertice $u$ est\'e conectado consigo mismo,
    la relaci\'on \begin{center}
      \texttt{$u$ est\'a conectado con $v$}

    \end{center}
    es una relaci\'on de equivalencia en el conjunto de v\'ertices del grafo $G,$
    \pause y las clases de equivalencia de esta relaci\'on son las componentes
    conexas de $G.$
  \end{rem}

\end{frame}

\begin{frame}{Distancia y diametro}
  Consideremos un grafo conexo $G.$ La distancia entre  dos v\'ertices $u$ y $v$
  en $G,$ denotada por $d(u,v),$ es la longitud del camino m\'as corto entre $u$
  y $v.$ E\~n diametro de $G,$ escrito $diam(G),$ es la distancia m\'axima entre
  cualesquiera dos puntos en $G.$
\end{frame}

\begin{frame}
  \begin{figure}[h!]
    \centering
    \includegraphics[width=10cm,keepaspectratio=true]{./grafo_8_9.png}
    % grafo_8_9.png: 0x0 pixel, 300dpi, 0.00x0.00 cm, bb=
    \caption{Distancia y diametro}
    \label{fig:md0505}
  \end{figure}
  Por ejemplo, en el grafo \ref{fig:md0505}(a), el diamtero es $3,$ mientras que
  en el (b), el diametro es 4.
\end{frame}

\begin{frame}{Puntos de corte y puentes}
  Sea $G$ un grafo conexo. Un v\'ertice $v$ en $G$ es llamado \emph{punto de
    corte} si $G-v$ es disconexo. Una arista $e$ en $G$ es llamada \emph{puente} si
  $G-e$ es disconexo.

  \begin{figure}[h]
    \centering
    \includegraphics[height=3cm,keepaspectratio=true]{./fig0809.png}
    % fig0809.png: 0x0 pixel, 300dpi, 0.00x0.00 cm, bb=
    \caption{Puntos de corte y puentes}
    \label{fig:0809}
  \end{figure}

\end{frame}

\subsection{Grafos transitables y eulerianos}
\begin{frame}
  \begin{figure}[h!]
    \centering
    \includegraphics[width=10cm,keepaspectratio=true]{./grafo_8_10.png}
    % grafo_8_10.png: 0x0 pixel, 300dpi, 0.00x0.00 cm, bb=
    \caption{Puentes de K\"onigsberg y su representaci\'on}
    \label{fig:md0506}
  \end{figure}
\end{frame}

\begin{frame}
  Un multigrafo es llamado \emph{transitable} si existe un \emph{paseo} (un
  camino d\'onde todos las aristas son diferentes), que incluye \emph{todos los
    v\'ertices y todas las aristas.}

  Tal paseo ser\'a llamado  \emph{paseo transitable}.
  \pause

  \begin{rem}
    De manera equivalente, un paseo transitable es un camino en el que todos los
    v\'ertices se transitan \emph{al menos} una vez,  pero las aristas
    \emph{exactamente} una vez.
  \end{rem}

\end{frame}

\begin{frame}

  \begin{prop}
    Cualquier grafo conexo y finito con exactamente dos v\'ertices impares es
    transitable. Un paseo transitable puede comenzar en alguno de los v\'ertices
    impares y terminar en el otro v\'ertice impar.
  \end{prop}
\end{frame}

\begin{frame}
  Un grafo $G$ es llamado \emph{grafo Euleriano} si existe un \emph{paseo
    transitable cerrado}.

  \pause
  A tal paseo le llamaremos \emph{paseo Euleriano.}
  \pause

  \begin{thm}[Euler]
    Un grafo conexo y finito es Euleriano si y solo si cada v\'ertice tiene grado
    par.
  \end{thm}

\end{frame}

\begin{frame}{Grafos hamiltonianos}
  En la definici\'on de grafos Eulerianos se enfatiz\'o pasar por todas las
  aristas.
  \pause

  Ahora, nos enfocaremos en visitar todos los v\'ertices.
\end{frame}

\begin{frame}
  Un \emph{circuito Hamiltoniano} es un grafo $G$ es un camino cerrado que visita
  cada v\'ertice en $G$ \emph{exactamente} una vez.
  \pause

  Si $G$ admite un circuito Hamiltoniano, entonces $G$ es llamado un \emph{grafo
    Hamiltoniano.}

  \pause
  \begin{rem}
    En la definici\'on de circuito Hamiltoniano, cuando decimos que el camino
    \emph{visita} cada v\'ertice exactamente una vez significa que, aunque el
    v\'ertice inicial tiene que ser el mismo que el final, todos los dem\'as
    v\'ertices intermedios deben ser distintos.
  \end{rem}

\end{frame}

\begin{frame}
  \begin{rem}
    Un {\color{red}paseo Euleriano} atraviesa {\color{red}cada una de las aristas}
    exactamente una vez, pero los v\'ertices se pueden repetir, mientras que un
      {\color{blue}circuito Hamiltoniano} visita {\color{blue}cada uno de los
        v\'ertices} exactamente una vez, pero las aristas pueden repetirse.

  \end{rem}

  \pause

  \begin{thm}
    Sea $G$ un grafo conexo con $n$ v\'ertices. Entonces $G$ es Hamiltoniano si
    $n\geq 3$ y $n \leq \deg(v)$ para cada v\'ertice $v$ en $G.$
  \end{thm}

\end{frame}

\begin{frame}
  \begin{figure}[h!]
    \centering
    \includegraphics[width=10 cm]{./grafo_8_11.png}
    % grafo_8_11.png: 0x0 pixel, 300dpi, 0.00x0.00 cm, bb=
    \caption{Circuitos Eulerianos y Hamiltonianos}
    \label{fig:md 0506}
  \end{figure}

\end{frame}

\subsection{Matriz de adyacencia}

\begin{frame}
  Supongamos que $G$ es un gr\'afo con $m$ v\'ertices y que estos han sido
  ordenados:
  $$
    v_{1}, v_{2},...,v_{m}.
  $$

  Entonces, la \emph{matriz de adyacencia} $A=\left( a_{i,j} \right)$ del grafo
  $G$ es la matriz de dimensi\'on $m\times m$ definida por:
  $$a_{i,j}=
    \begin{cases}
      1 & v_{i}\texttt{ es adyacente a }v_{j} \\
      0 & \texttt{en otro caso}
    \end{cases}
  $$
\end{frame}

\begin{frame}
  \begin{figure}[h]
    \centering
    \includegraphics[width=10cm,keepaspectratio=true]{./fig0827.png}
    % fig0827.png: 0x0 pixel, 300dpi, 0.00x0.00 cm, bb=
    \caption{Matriz de adyacencia}
    \label{fig:0827}
  \end{figure}

\end{frame}

\section{Digrafos}

\begin{frame}
  Los \emph{grafos dirigidos} o \emph{digrafos} son grafos en los que las aristas
  tienen una direcci\'on.
\end{frame}

\subsection{Grafos dirigidos}

\begin{frame}
  Un grafo dirigido $G=G(V,E)$ consiste de:
  \begin{enumerate}
    \item Un conjunto $V=V(G)$ cuyos elementos son llamados \emph{v\'ertices};
    \item un conjunto $E=E(G)$ de \emph{pares ordenados} ordenados de v\'ertices
          llamados \emph{arcos} o \emph{aristas dirigidas.}
  \end{enumerate}

\end{frame}

\begin{frame}
  Supongamos que $e=(u,v)$ es un arco en el digrafo $G.$ Entonces, la siguiente
  terminología es usada:
  \begin{itemize}
    \item $e$ comienza en $v$ y termina en $v;$
    \item $u$ es el origen o punto inicial de $e,$ mientras que $v$ es el destino
          o punto final de $e.$
    \item $v$ es un sucesor de $u;$
    \item $u$ es adyacente a $v$ y $v$ es adyacente desde $u.$
  \end{itemize}
  \pause

  Si $u=v,$ $e$ es llamado un \emph{bucle.}
\end{frame}

\begin{frame}
  Si las aristas o los v\'ertices de un digrafo est\'an etiquetas con alg\'un
  tipo de dato, diremos que es un \emph{digrafo etiquetado.}
  \pause

  De manera similar a un grafo, un digrafo ser\'a finito si el conjunto de
  v\'ertices y el de aristas es finito.
\end{frame}

\begin{frame}
  \begin{exmp}
    Consideremos el siguiente digrafo.
    \begin{figure}[h]
      \centering
      \includegraphics[height=3cm,keepaspectratio=true]{./fig0901a.png}
      % fig0901a.png: 0x0 pixel, 300dpi, 0.00x0.00 cm, bb=
      %\caption{Digrafo}
      \label{fig:0901a}
    \end{figure}
    Las aristas $e_{2}$ y $e_{3}$ son llamados \emph{paralelos,} ya que ambos
    comienzan en $B$ y terminan en $A.$ La arista $e_{7}$ es un \emph{bucle.}
  \end{exmp}

\end{frame}

\begin{frame}
  \begin{exmp}
    \begin{figure}[h]
      \centering
      \includegraphics[width=5cm,keepaspectratio=true]{./fig0901b.png}
      % fig0901b.png: 0x0 pixel, 300dpi, 0.00x0.00 cm, bb=
      \caption{Proceso estocástico}
      \label{fig:0901b}
    \end{figure}

  \end{exmp}

\end{frame}

\subsection{Matriz de adyacencia}

\begin{frame}
  Ahora, s\'olo consideraremos \emph{digrafos simples} $G(V,E)$, es decir, sin
  aristas paralelas. Entonces $E$ es simplemente una relaci\'on en $V.$ \pause

  De manera inversa, si $R$ es una relaci\'on en $V,$ entonces $G(V,R)$ es un
  digrafo simple.\pause

  En unidades anteriores, ya hemos construido digrafos asociados a relaciones de
  orden parcial, llamados diagramas de Hasse.
\end{frame}

\begin{frame}
  Supongamos que $G$ es un digrafo simple con $m$ v\'ertices, y supongamos que
  los v\'ertices de $G$ han sido ordenados y son llamados $v_{1},
    v_{2},...,v_{m}.$ \pause

  Entonces la \emph{matrix de adyacencia} $A=\left( a_{i,j} \right)$ de $G$ es la
  una matriz de dimensi\'on $m\times m$ definida de la siguiente manera
  $$a_{i,j}=
    \begin{cases}
      1 & \exists e \in E: e=(v_{i}, v_{j}) \\
      0 & \texttt{en otro caso}
    \end{cases}
  $$
\end{frame}

\begin{frame}
  \begin{rem}
    Las matrices de adyacencia de un mismo grafo dependen del orden en que se
    enumeren los v\'ertices. \pause
    Sin embargo, dos matrices de adyacencia de un mismo grafo est\'an
    relacionadas por operaciones elementales: cambiar el orden de columnas y
    renglones.
  \end{rem}

\end{frame}

\begin{frame}
  \begin{exmp}
    Sea $G$ el siguiente digrafo
    \begin{figure}[h]
      \centering
      \includegraphics[height=3cm,keepaspectratio=true]{./fig0904a.png}
      % fig0904a.png: 0x0 pixel, 300dpi, 0.00x0.00 cm, bb=
      \caption{Construya su matriz de adyacencia del digrado anterior.}
      \label{fig:0904a}
    \end{figure}

  \end{exmp}

\end{frame}

\begin{frame}
  La matriz identidad $I_{m}=\left( I_{i,j} \right)$ de dimensi\'on $m\times m$
  se define como
  $$I_{i,j}
    \begin{cases}
      1 & i=j       \\
      0 & i \neq j,
    \end{cases}
  $$\pause
  es decir, es matriz cuadrangular con $1's$ en la \emph{diagonal principal}, y
  ceros en cualquier otra entrada.
\end{frame}

\begin{frame}
  \begin{exmp}
    $$I_{2}=
      \begin{pmatrix}
        1 & 0 \\
        0 & 1 \\
      \end{pmatrix}
    $$

    $$I_{3}=
      \begin{pmatrix}
        1 & 0 & 0 \\
        0 & 1 & 0 \\
        0 & 0 & 1
      \end{pmatrix}
    $$
  \end{exmp}

\end{frame}

\begin{frame}
  La propiedad principal de una matriz identidad $I_{m}$ es que es nuestra
  respecto a la multiplicaci\'on de matrices\pause, es decir, para cualquier otra
  matriz $A\in M_{n}:$
  $$
    AI_{n}=I_{n}A=A.
  $$
\end{frame}

\begin{frame}
  La potencia $n-$\'esima de una matriz $A \in M_{n}$ se define de manera
  recursiva como
  $$
    A^{n}=
    \begin{cases}
      I_{n}    & n=0       \\
      AA^{n-1} & n \in \N.
    \end{cases}
  $$ \pause

  Es decir, $$A^{0}=I, A^{1}=A, A^{2}=AA,...$$
\end{frame}

\begin{frame}
  Definamos $a_{k}(i,j)$ como la entrada en la posici\'on $i,j$ de $A^{k}.$
  \pause

  \begin{prop}
    Sea $A$ la matriz de adyacencia de un grafo $G.$ Entonces $a_{k}(i,j)$ es
    igual al n\'umero de caminos de longitud $k$ que van de $v_{i}$ a $v_{j}.$
  \end{prop}

\end{frame}

\begin{frame}{Ejemplo}
  Consideremos nuevamente el grafo
  \begin{figure}[h]
    \centering
    \includegraphics[height=3cm,keepaspectratio=true]{./fig0904a.png}
    % fig0904a.png: 0x0 pixel, 300dpi, 0.00x0.00 cm, bb=
  \end{figure}
\end{frame}

\begin{frame}
  Recordemos que su matriz de adyacencia es
  \begin{equation}
    \label{exmp:adj}
    \tag{AD}
    A= \left(\begin{array}{rrrr}
        0 & 0 & 0 & 1 \\
        1 & 0 & 1 & 1 \\
        1 & 0 & 0 & 1 \\
        1 & 0 & 1 & 0
      \end{array}\right)
  \end{equation}

\end{frame}

\begin{frame}
  Entonces
  $$
    A^{2}= \left(\begin{array}{rrrr}
        1 & 0 & 1 & 0 \\
        2 & 0 & 1 & 2 \\
        1 & 0 & 1 & 1 \\
        1 & 0 & 0 & 2
      \end{array}\right) \;
    A^{3}= \left(\begin{array}{rrrr}
        1 & 0 & 0 & 2 \\
        3 & 0 & 2 & 3 \\
        2 & 0 & 1 & 2 \\
        2 & 0 & 2 & 1
      \end{array}\right) \;
  $$

  $$
    A^{4}= \left(\begin{array}{rrrr}
        2 & 0 & 2 & 1 \\
        5 & 0 & 3 & 5 \\
        3 & 0 & 2 & 3 \\
        3 & 0 & 1 & 4
      \end{array}\right)
  $$
\end{frame}

\begin{frame}
  Observe que $a_{2}(4,1)=1,$ de manera que existe un solo camino de longitud 2
  de $v_{4}$ a $v_{1}.$ De manera similar, como $a_{3}(2,3)=2,$ entonces existen
  dos caminos de longitud $3$ de $v_{2}$ a $v_{3}.$
\end{frame}

\begin{frame}
  \begin{rem}
    Si definimos
    $$B_{r}= \sum_{i=1}^{r}A^{i},$$
    entonces la entrada $i,j$ de esta matriz nos indicar\'a el n\'umero de caminos
    de longitud a lo m\'as $r$ de $v_{i}$ a $v_{j}.$
  \end{rem}
\end{frame}

\begin{frame}
  En nuestro ejemplo, considerando $A$ dado por \eqref{exmp:adj}, tenemos que
  \begin{equation}
    \label{B4}
    B_{4}=
    \left(\begin{array}{rrrr}
        4  & 0 & 3 & 4  \\
        11 & 0 & 7 & 11 \\
        7  & 0 & 4 & 7  \\
        7  & 0 & 4 & 7
      \end{array}\right)
  \end{equation}
  \pause

  ?`Existe alguna manera de llegar al vertice $v_{2}$ desde el v\'ertice $v_{1}$,
  sin importar la longitud del camino?
\end{frame}

\subsection{Matriz de accesibilidad}

\begin{frame}
  Sea $G=G(V,E)$ un grafo simple dirigido con $m$ v\'ertices $v_{1},...,v_{m}.$
  La \emph{matriz de accesibilidad} de $G$ es la matriz $m-$cuadrangular
  $P=\left( p_{ij} \right)$ definida de la siguiente manera:
  $$p_{ij}=
    \begin{cases}
      1 & \texttt{existe un camino de }v_{i}\texttt{ a }v_{j} \\
      0 & \texttt{en otro caso}
    \end{cases}
  $$
\end{frame}

\begin{frame}
  \begin{prop}
    Sea $A$ la matriz de adyacencia de un grafo $G$ con $m$ v\'ertices. Entonces
    la matriz de accesibilidad y
    \begin{equation}
      \label{Bm}
      B_{m}=\sum_{i=1}^{m}A^{i}
    \end{equation}
    tienen exactamente las mismas entradas no nulas.
  \end{prop}

\end{frame}

\begin{frame}
  \begin{defn}
    Un digrafo es \emph{fuertemente conexo} si para cualquier par de v\'ertices
    $u,v$ existe al menos un camino de $u$ a $v$ y otro de $v$ a $u.$
  \end{defn}

\end{frame}

\begin{frame}
  \begin{prop}
    Sea $A\in M_{m}$ la matriz de adyacencia de un grafo $G.$ Entonces, las
    siguientes proposiciones son equivalentes:
    \begin{enumerate}
      \item $G$ es fuertemente conexo;
      \item la matriz de accesibilidad $P$ no tiene entradas nulas;
      \item la matriz $B_{m},$ dada por \eqref{Bm}, no tiene entradas nulas.
    \end{enumerate}

  \end{prop}

\end{frame}

\begin{frame}
  \begin{exmp}
    \label{lip:exmp:0908}
    Para encontrar la matriz de accesibilidad asociada a la matriz de adyacencia
    $A,$ dada por \eqref{exmp:adj}, basta sustitur las entradas no nulas en la
    matriz $B_{4},$ dada por \eqref{B4}, por $1's:$
    $$
      P=
      \left(\begin{array}{rrrr}
          1 & 0 & 1 & 1 \\
          1 & 0 & 1 & 1 \\
          1 & 0 & 1 & 1 \\
          1 & 0 & 1 & 1
        \end{array}\right)
    $$
  \end{exmp}

\end{frame}

\end{document}
