\documentclass[xcolor={svgnames},
  %hyperref={colorlinks,citecolor=Blue,linkcolor=Blue,urlcolor=DarkBlue}, 
  hyperref={colorlinks}, 
  spanish, 12pt]{beamer}
  \mode<presentation>
  
  \usefonttheme[onlymath]{serif}
  \setbeamertemplate{theorems}[ams style] 
  
  
\usetheme{metropolis}
 \useinnertheme{rectangles} 
\setbeamertemplate{navigation symbols}{}
%\setbeamertemplate{caption}[numbered]
%\useoutertheme{infolines}
\usepackage{cleveref}

\setbeamercovered{highly dynamic}

\newcounter{saveenumi}
\newcommand{\seti}{\setcounter{saveenumi}{\value{enumi}}}
\newcommand{\conti}{\setcounter{enumi}{\value{saveenumi}}}

\resetcounteronoverlays{saveenumi}

%\usepackage{beamerthemebars}
\usepackage{fontenc}
\usepackage{graphicx}
\usepackage[utf8]{inputenc}
\usepackage[spanish,mexico]{babel}
\usepackage{fontenc}
\usepackage{amsmath}
\usepackage{amsthm}
\usepackage{amssymb}
\usepackage{graphicx}
\usepackage{mathrsfs}
\usepackage{yfonts}
%\usepackage{hyperref}
\usepackage{enumerate}
\usepackage{mathtools}
\usepackage{textcomp}
\usepackage{lmodern}
\usepackage{fancyvrb}
\usepackage{multicol}

\DefineVerbatimEnvironment{ColorVerbatim}{Verbatim}%
  {formatcom=\color{purple},commandchars=\\\{\}}
  
\usepackage{etoolbox}

\BeforeBeginEnvironment{Verbatim}{\begingroup\color{purple}}%
\AfterEndEnvironment{Verbatim}{\endgroup}%
%\usepackage{etoolbox}
% \AtBeginEnvironment{enumerate}{\begin{multicols}{2}}    %%% this line
% \AtEndEnvironment{enumerate}{\end{multicols}}            %%% and this one
% \usepackage{multicol}
%\usepackage[usenames,dvipsnames,svgnames,table]{xcolor}
%\usepackage[urlcolor=blue]{hyperref}
%\numberwithin{section}{part}
\numberwithin{equation}{section} %% Comment out for sequentially-numbered
\numberwithin{figure}{section} %% Comment out for sequentially-numbered

% Automatically generate section title slides in beamer?
% Add support for \subsubsectionpage
\def\subsubsectionname{\translate{Subsubsection}}
\def\insertsubsubsectionnumber{\arabic{subsubsection}}
\setbeamertemplate{subsubsection page}
{
  \begin{centering}
    {\usebeamerfont{subsubsection name}\usebeamercolor[fg]{subsubsection name}\subsubsectionname~\insertsubsubsectionnumber}
    \vskip1em\par
    \begin{beamercolorbox}[sep=4pt,center]{part title}
      \usebeamerfont{subsubsection title}\insertsubsubsection\par
    \end{beamercolorbox}
  \end{centering}
}
\def\subsubsectionpage{\usebeamertemplate*{subsubsection page}}

\AtBeginSection{\frame{\sectionpage}}
\AtBeginSubsection{\frame{\subsectionpage}}
\AtBeginSubsubsection{\frame{\subsubsectionpage}}

\theoremstyle{plain}
  \newtheorem{thm}{Teorema}[section]
  \newtheorem{prop}{Proposici\'on}[section]
  \newtheorem{lem}[thm]{Lema}
  \newtheorem{cor}[thm]{Corolario}
  %\newtheorem{rem}{Observaci\'on}[chapter]
  \newtheorem*{sol}{Soluci\'on}
  \newtheorem{alg}{Algoritmo}[section]

\theoremstyle{definition}
  \newtheorem{defn}{Definici\'on}[section]
  \newtheorem{conj}{Conjectura}[section]
  \newtheorem{exmp}{Ejemplo}[section]
  \newtheorem{exe}{Ejercicio}[section]
  \newtheorem{solved}{Ejercicio Resuelto}[section]
  \newtheorem{evc}{Evaluaci\'on Continua}[section]
  \newtheorem{prob}{Problema}[section]
  \newtheorem{rem}{Observaci\'on}[section]
  \newtheorem*{ax}{Axioma}

\theoremstyle{remark}
  \newtheorem{claim}{Afirmaci\'on}[section]
  %\newtheorem{rem}{Observaci\'on}[chapter]
  %\newtheorem*{note}{Nota}
  \newtheorem{case}{Caso}
  \newtheorem{hint}{Sugerencia}[section]

\input{./00-01-comandos.tex}

%\date{\today}
\title{Matem\'aticas Discretas}
\author[Juliho Castillo]{\href{https://www.youtube.com/channel/UCb1i-EtybaWWX5urFfmMUWQ}{M. en C. Juliho Castillo}}

\institute[ITESM CMM]{Tec de Monterrey, Campus Ciudad de M\'exico}
\date{\today}

\begin{document}

\logo{
 \includegraphics[width=3cm,keepaspectratio=true]{./logo.png}
 %LOGO-CECYTEO.png: 200x200 pixel, 72dpi, 7.06x7.06 cm, bb=0 0 200 200
}

\frame{
\titlepage
}

\frame{\tableofcontents}

% \AtBeginSubsection[]
% {
%   \begin{frame}
%     \tableofcontents[currentsubsection]
%   \end{frame}
% }

%\part{Presentaci\'on}

\section{Acerca de m\'i}

\begin{frame}
 \frametitle{!`Bienvenidos a Matem\'aticas Discretas!}
 Mi nombre es Juliho Castillo... \pause
 (con ``h'' entre la ``i'' y la ``o'') \pause
 
 Mi correo es \href{mailto:jdcastillo@comunidad.unam.mx}{jdcastillo@comunidad.unam.mx}
\end{frame}

\begin{frame}[<+->]
 \frametitle{Educaci\'on}
 \begin{enumerate}
  \item[2017] \emph{Candidato a Doctor:} Instituto de Matemáticas, UNAM, posgrado en Ciencias Matemáticas. 
\item[2013] \emph{Maestro en Ciencias:} Departamento de Matemáticas, CINVESTAV, especialidad en Matemáticas.
\item[2011] \emph{Licenciado en Ciencias:} Escuela de Ciencias, UABJO, especialidad en Matemáticas.
 \end{enumerate}
\end{frame}

\begin{frame}
 \frametitle{Publicaciones}
 \begin{enumerate}
  \item[2016] \emph{Geometric and viscosity solutions for the Cauchy problem of first order:} ArXiv,
https://arxiv.org/abs/1611.10293, Preprint.
  \item[2014] \emph{Symplectic capacities on surfaces:} Manuscripta Mathematica, Vol. 229, artículo 701, Artículo de Investigación. En colaboración con Dr. Rystam Sadykov
  \item[2012] \emph{Aplicaciones del Control Estocástico al Análisis Semiclásico:} Aportaciones Matemáticas, Memorias 45, 69-96, Artículo de exposición.
 \end{enumerate}
\end{frame}

\begin{frame}
 \frametitle{Becas y Reconocimientos}
 \begin{enumerate}
  \item[2013-17] Becario CONACYT, Programa de Doctorado (Nivel Internacional)
  \item[2011-13] Becario CONACYT, Programa de Maestría (Nivel internacional)
  \item[2011] Premio Nacional “Mixbaal” a las Mejores Tesis de Licenciatura en Matemáticas Aplicadas, Mención Honorífica
\item[2011] Conferencista invitado, Sesión del Premio Mixbaal, ENOAN XXI
\item[2001] Seleccionado Estatal, XV Olimpiada Mexicana de Matemáticas, Delegación Oaxaca

 \end{enumerate}

\end{frame}

\begin{frame}
 \frametitle{Experiencia docente}
 \begin{itemize}
\item Profesor de Asignatura (Universidad Panamericana, Universidad Anáhuac México Sur, Instituto Blaise Pascal)
\item Cursos independientes (ITO, CECYTEO, Club de Matemáticas “Teorema”)
\item Entrenador de la Olimpiada Mexicana de Matemáticas, Delegación Oaxaca.
\end{itemize}

\end{frame}

%\part{Syllabus}
\section{Informaci\'on General}

\begin{frame}
 \frametitle{Intenci\'on del curso}
 Curso de nivel básico de computación en el que los alumnos utilizarán herramientas matemáticas para modelar situaciones
de la vida real, como los conjuntos, relaciones y funciones, cálculo de proposiciones, cálculo de predicados y teoría de
grafos. \pause 

Requiere de conocimientos previos de matemáticas básicas del nivel preparatoria. 
\pause

Como resultados del aprendizaje
se alcanzar\'a la presentación formal y rigurosa de problemas de computación y electrónica y su posible solución, uso de razonamiento
formal.
\end{frame}

\begin{frame}
 \frametitle{Objetivo general}
 Al nalizar el curso, el alumno será capaz de: razonar de manera rigurosa y formal usando las herramientas que le da la
lógica formal, usar la inducción como método de razonamiento y demostración y usar los grafos como medio de modelar
estructuras de datos.

\end{frame}


\begin{frame}
 \frametitle{Temario}

 \begin{enumerate}
  \item Lógica Matemática \pause
  \item Teoría de conjuntos e inducción matemática \pause
  \item Relaciones y Funciones \pause
  \item Grafos
 \end{enumerate}
\end{frame}

\begin{frame}
 \frametitle{Lógica Matemática}
 Identificar\'as y escribir\'as expresiones bien formadas en Lógica Proposicional y de predicados para determinar la validez de argumentos en la interpretaci\'on de situaciones de la vida real.
\end{frame}

\begin{frame}
 \frametitle{Teoría de conjuntos e inducción matemática}
 \begin{enumerate}
  \item  Identificar\'as problemas donde la teoría de conjuntos puede ser aplicada.
 \pause
  \item Aplicar\'as las operaciones de la teoría de conjuntos en la solución de problemas. \pause
  \item Aplicar\'as correctamente los principios de inducción matemática a situaciones en el conjunto de los números naturales.

 \end{enumerate}

\end{frame}

\begin{frame}
 \frametitle{Relaciones y Funciones}
 Describir\'as un problema o una situación determinando los elementos a considerar y las estructuras con que mejor pueden
representarse, y después elabora la descripción seleccionando correctamente entre una función o relación.

\end{frame}

\begin{frame}
 \frametitle{Grafos}
 Abstraer\'as situaciones simples de la electrónica y la computación por medio de grafos. Usa propiedades de grafos para
describir problemas en el modelo.

\end{frame}


\begin{frame}
 \frametitle{Bibliografía}
\begin{enumerate}
 \item LIPSCHUTZ, S., LIPSON, M.; ``Discrete Mathematics''; Schaum’s Outline Series, McGraw-Hill, 3th Edition, 2007. \pause
 \item Epp, Susanna S., Discrete mathematics with applications, 3rd ed., California : Belmont, CA : Thomson-Brooks/Cole, eng, 
0534359450 (papel no ácido), 0534490964 (ed. de estudiante internacional), 9780534359454 (papel no ácido),
\item  Johnsonbaugh, R, Discrete Mathematics, 6, Pearson Prentice Hall, 2009, 
\item Grimaldi, R. P, Discrete and Combinatorial Mathematics, 5, Addison Wesley, 2013
\item  Rosen, Kenneth H., Discrete mathematics and its applications, 6th ed., Massachusetts : Boston : McGraw-Hill, 2007, eng, 
0071244743, 0072880082, 9780071244749, 9780072880083
\end{enumerate}
\end{frame}

\begin{frame}
 \frametitle{Normativa}

La calificación final se compondrá de la siguiente manera:
\begin{enumerate}
 \item 2 evaluaciones parciales x 30\% cada una = 60\% calificación final
 \item 1 examen final = 40\% calificación final
\end{enumerate}

\pause 

Cada evaluación parcial se compondrá de la siguiente manera
\begin{enumerate}
 \item Examen escrito = 70\%
 \item Gu\'ia de examen = 10\%
 \item Evaluaci\'on continua = 20\%
\end{enumerate}

\end{frame}

\begin{frame}
Algunos puntos importantes a considerar son:

1.) El límite de faltas es el 20\% del total de las sesiones y no hay justificante para las mismas.

2.) En clase, no se podrá hacer uso de dispositivos electrónicos, excepto cuando la actividad lo requiera y el profesor lo indique.

\end{frame}
\end{document}
