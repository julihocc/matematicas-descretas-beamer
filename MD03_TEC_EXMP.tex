\documentclass[xcolor={svgnames},
  %hyperref={colorlinks,citecolor=Blue,linkcolor=Blue,urlcolor=DarkBlue}, 
  hyperref={colorlinks},
  %handout,
  t,
  spanish, 12pt]{beamer}
  \mode<presentation>
  
  \usefonttheme[onlymath]{serif}
  \setbeamertemplate{theorems}[ams style] 
  
  \addtolength{\headsep}{1cm}
  
\usetheme{metropolis}
 \useinnertheme{rectangles} %\setbeamertemplate{navigation symbols}{}
%\setbeamertemplate{caption}[numbered]
%\useoutertheme{infolines}
%\usepackage{cleveref}

\metroset{block=fill}
%\usecolortheme{beaver}
\usecolortheme{dolphin}
%\usecolortheme{whale}

\setbeamercovered{highly dynamic}

\newcounter{saveenumi}
\newcommand{\seti}{\setcounter{saveenumi}{\value{enumi}}}
\newcommand{\conti}{\setcounter{enumi}{\value{saveenumi}}}

\resetcounteronoverlays{saveenumi}

%\pagestyle{empty}
\setbeamercolor{emph}{fg=purple}
\renewcommand<>{\emph}[1]{%
  {\usebeamercolor[fg]{emph}\only#2{\itshape}#1}%
}

%\usepackage{beamerthemebars}
\usepackage{fontenc}
\usepackage{graphicx}
\usepackage[utf8]{inputenc}
\usepackage[spanish,mexico]{babel}
\usepackage{fontenc}
\usepackage{amsmath}
\usepackage{amsthm}
\usepackage{amssymb}
\usepackage{graphicx}
\usepackage{mathrsfs}
\usepackage{yfonts}
%\usepackage{hyperref}
\usepackage{enumerate}
\usepackage{mathtools}
\usepackage{textcomp}
\usepackage{lmodern}
\usepackage{fancyvrb}
\usepackage{multicol}
\usepackage{color}
\usepackage{verbatim}

\DefineVerbatimEnvironment{ColorVerbatim}{Verbatim}%
  {formatcom=\color{purple},commandchars=\\\{\}}
  
\usepackage{etoolbox}

\BeforeBeginEnvironment{Verbatim}{\begingroup\color{purple}}%
\AfterEndEnvironment{Verbatim}{\endgroup}%
%\usepackage{etoolbox}
% \AtBeginEnvironment{enumerate}{\begin{multicols}{2}}    %%% this line
% \AtEndEnvironment{enumerate}{\end{multicols}}            %%% and this one
% \usepackage{multicol}
%\usepackage[usenames,dvipsnames,svgnames,table]{xcolor}
%\usepackage[urlcolor=blue]{hyperref}
%\numberwithin{section}{part}
\numberwithin{equation}{section} %% Comment out for sequentially-numbered
\numberwithin{figure}{section} %% Comment out for sequentially-numbered

% Automatically generate section title slides in beamer?
% Add support for \subsubsectionpage
\def\subsubsectionname{\translate{}}
\def\insertsubsubsectionnumber{\arabic{subsubsection}}
\setbeamertemplate{subsubsection page}
{
  \begin{centering}
    {\usebeamerfont{subsubsection name}\usebeamercolor[fg]{subsubsection name}\subsubsectionname}%~\insertsubsubsectionnumber}
    \vskip1em\par
    \begin{beamercolorbox}[sep=4pt,center]{part title}
      \usebeamerfont{subsubsection title}\insertsubsubsection\par
    \end{beamercolorbox}
  \end{centering}
}
\def\subsubsectionpage{\usebeamertemplate*{subsubsection page}}

\AtBeginSection{\frame{\sectionpage}}
\AtBeginSubsection{\frame{\subsectionpage}}
\AtBeginSubsubsection{\frame{\subsubsectionpage}}

\newtheorem{exmp}{Ejemplo}
%\input{./00-03-entorno.tex}
\input{./00-01-comandos.tex}

%\date{\today}
\title{Ejemplos de Inducción y Recursión}
\author[Juliho Castillo]{\href{https://www.youtube.com/channel/UCb1i-EtybaWWX5urFfmMUWQ}{M. en C. Juliho Castillo}}

\institute[ITESM-CCM]{Tec de Monterrey, Campus Ciudad de M\'exico}
\date{\today}

\begin{document}

\logo{
 \includegraphics[width=1cm,keepaspectratio=true]{./logo.png}
 %LOGO-CECYTEO.png: 200x200 pixel, 72dpi, 7.06x7.06 cm, bb=0 0 200 200
}

\frame{
\titlepage
}

%\begin{frame}[allowframebreaks=0.5]
% \tableofcontents
%\end{frame}

%\frame{\tableofcontents}

% \AtBeginSubsection[]
% {
%   \begin{frame}
%     \tableofcontents[currentsubsection]
%   \end{frame}
% }
\begin{frame}
\begin{exmp}
	Demuestre por inducción que 
	$\displaystyle 2+4+6+...+2n=n(n+1)$
\end{exmp}
\end{frame}

\begin{frame}
\begin{exmp}
	Demuestre por inducción que 
$\displaystyle 1+4+7+...+\left( 3n-2 \right)=\dfrac{n\left( 3n-1 \right)}{2}$
\end{exmp}
\end{frame}

\begin{frame}
\begin{exmp}
	Demuestre por inducción que 
	$\displaystyle 1^{2}+2^{2}+...+n^{2}=\dfrac{n(n+1)(2n+1)}{6}$
\end{exmp}
\end{frame}

\begin{frame}
\begin{exmp}
	Demuestre por inducción que 
 $\displaystyle \dfrac{1}{1\cdot 3}+\dfrac{1}{3\cdot 5}+...+\dfrac{1}{\left( 2n-1 \right)\cdot \left( 2n+1 \right)}=\dfrac{n}{2n+1}$
\end{exmp}
\end{frame}

\begin{frame}
	\begin{exmp}
	Demuestre por inducción que 		
$\displaystyle \dfrac{1}{1\cdot 5}+ \dfrac{1}{5 \cdot 9}+...+\dfrac{1}{(4n-3)\cdot (4n+1)}=\dfrac{n}{4n+1}$
	\end{exmp}
\end{frame}

\begin{frame}
	\begin{exmp}
	Demuestre por inducción que 		
 $7^{n}-2^{n}$ es divisible entre $5$
	\end{exmp}
\end{frame}

\begin{frame}
	\begin{exmp}
		Demuestre por inducción que 		
$n^{3}-4n+6$ es divisible entre $3$
	\end{exmp}
\end{frame}

\begin{frame}
	\begin{exmp}
	 La funci\'on de Ackermann est\'a definida de manera recursiva de las siguiente manera:
$$
A(m,n)=
\begin{cases}
n+1 & m=0\\
A(m-1,1) & m\neq0, n=0 \\
A(m-1, A(m,n-1)) & m\neq 0, n\neq 0
\end{cases}
$$
Encuentre $A(1,1)$.		
	\end{exmp}

\end{frame}
\end{document}